\documentclass[a4paper,11pt]{article}
\usepackage{amsmath,amsfonts,amssymb,amsthm}
\usepackage{graphicx}
\usepackage{fullpage}
\usepackage{caption}
\usepackage{setspace}
\usepackage{hyperref}
\usepackage{enumerate}
\usepackage[all]{xy}
\usepackage[margin=1in]{geometry}
\usepackage{multirow}
\usepackage{bm}
\usepackage[toc,page]{appendix}
\usepackage{geometry}
\usepackage{siunitx}

\usepackage{listings}
\usepackage{color} %red, green, blue, yellow, cyan, magenta, black, white
\definecolor{mygreen}{RGB}{28,172,0} % color values Red, Green, Blue
\definecolor{mylilas}{RGB}{170,55,241}

\geometry{tmargin=0.7in,bmargin=0.7in,lmargin=0.9in,rmargin=0.9in}

\numberwithin{equation}{section}
\newtheorem{thm}{Theorem}[section]
\newtheorem{lem}[thm]{Lemma}
\newtheorem{cor}[thm]{Corollary}
\newtheorem{exa}[thm]{Example}
\newtheorem{prop}[thm]{Proposition}
\newtheorem{defn}[thm]{Definition}
\newtheorem{claim}[thm]{Claim}
\theoremstyle{remark}
\newtheorem*{rem}{Remark}


\newcommand{\Q}{\mathbb Q}
\newcommand{\Z}{\mathbb Z}
\newcommand{\N}{\mathbb N}
\newcommand{\R}{\mathbb R}
\newcommand{\C}{\mathbb C}
\newcommand{\HH}{\mathbb H}
\newcommand{\F}{\mathbb F}


\title{Reinforcement Learning: An Introduction \\ Attempted Solutions \\ Chapter 2}
\author{Scott Brownlie \& Rafael Rui}
\date{}


\begin{document}
%\pagenumbering{gobble}
\maketitle
%\newpage
%\pagenumbering{arabic}

\section{Exercise 2.1}

\textbf{In $\epsilon$-greedy action selection, for the case of two actions and $\epsilon = 0.5$, what is the probability that the greedy action is selected?}
\\ \\
The greedy action is initially selected with probability 0.5, and if not, then one of the two actions is selected randomly (each with probability 0.5). Therefore, the overall probability that the greedy action is selected is
\[
	0.5 + 0.5 \cdot 0.5 = 0.75.
\]


\section{Exercise 2.2: Bandit example}

\textbf{Consider a $k$-armed bandit problem with $k = 4$ actions, denoted 1, 2, 3, and 4. Consider applying to this problem a bandit algorithm using $\epsilon$-greedy action selection, sample-average action-value estimates, and initial estimates of $Q_1(a) = 0$, for all $a$. Suppose the initial sequence of actions and rewards is $A_1 = 1, R_1 = 1, A_2 = 2, R_2 = 1, A_3 = 2, R_3 = 2, A_4 = 2, R_4 = 2, A_5 = 3, R_5 = 0$. On some of these time steps the $\epsilon$ case may have occurred, causing an action to be selected at random. On which time steps did this definitely occur? On which time steps could this possibly have occurred?}
\\ \\
The action value estimates on each time step are as follows:
\begin{enumerate}
	\item $Q_1(1) = 0, Q_1(2) = 0, Q_1(3) = 0, Q_1(4) = 0$
	\item $Q_1(1) = 1, Q_1(2) = 0, Q_1(3) = 0, Q_1(4) = 0$
	\item $Q_1(1) = 1, Q_1(2) = 1, Q_1(3) = 0, Q_1(4) = 0$
	\item $Q_1(1) = 1, Q_1(2) = 3/2, Q_1(3) = 0, Q_1(4) = 0$
	\item $Q_1(1) = 1, Q_1(2) = 5/3, Q_1(3) = 0, Q_1(4) = 0$
	\item $Q_1(1) = 1, Q_1(2) = 5/3, Q_1(3) = 0, Q_1(4) = 0$
\end{enumerate}
The highest actions values on each time step were $\{1,2,3,4\}, \{1\}, \{1, 2\}, \{2\}, \{2\}$ and the chosen actions were $1,2,2,2,3$ respectively.
Therefore, the $\epsilon$ case definitely occurred on time steps 2 and 5. As it is possible that the greedy action is chosen randomly when the $\epsilon$ case occurs, the $\epsilon$ case could possibly have occurred on any of the remaining time steps.


\section{Exercise 2.3}

\textbf{In the comparison shown in Figure 2.2, which method will perform best in the long run in terms of cumulative reward and probability of selecting the best action? How much better will it be? Express your answer quantitatively.}
\\ \\ 
On each time step the probability of selecting the best action is $1-\epsilon$ times the probability that the greedy action is the best action, plus $\epsilon$ times the probability that the best action is chosen randomly. Clearly the probability that the best action is chosen randomly is $1/k$, thus we just need to work out the probability that the greedy action is the best action. 
\\ \\ 
On the first time step all actions have value 0 and we select an action $A_1$ at random and receive a reward $R_1$. Suppose that $R_1 < 0$. What is the probability that $A_1$ is the best action? 


\section{Exercise 2.4}

\textbf{If the step-size parameters, $\alpha_n$, are not constant, then the estimate $Q_n$ is a weighted average of previously received rewards with weighting different from that given by (2.6). What is the weighting on each prior reward for the general case, analogous to (2.6), in terms of the sequence of step-size parameters?}
\\ \\ 
We have
\begin{align*}
	Q_{n+1} & = Q_n + \alpha_n[R_n - Q_n] \\
			& = \alpha_n R_n + (1-\alpha_n) Q_n \\
			& = \alpha_n R_n + (1-\alpha_n)[\alpha_{n-1} R_{n-1} + (1-\alpha_{n-1}) Q_{n-1}] \\
			& = \alpha_n R_n + (1-\alpha_n)\alpha_{n-1}R_{n-1} + (1-\alpha_n)(1-\alpha_{n-1})Q_{n-1} \\
			& = \alpha_n R_n + (1-\alpha_n)\alpha_{n-1}R_{n-1} + \quad \dots + (1-\alpha_n)(1-\alpha_{n-1})\dots (1-\alpha_2)\alpha_1 R_1 + \\
			& \quad \quad (1-\alpha_n)(1-\alpha_{n-1})\dots (1-\alpha_2)(1-\alpha_1) Q_1 \\
			& = Q_1\prod_{j=1}^{n}(1-\alpha_j) + \alpha_n R_n + \sum_{i=1}^{n-1}\alpha_i R_i \prod_{j=i+1}^{n}(1-\alpha_j).
\end{align*}
Therefore, the weighting on $R_n$ is $\alpha_n$ and the weighting on $R_i$ for $i=1,\dots,n-1$ is $\alpha_i \prod_{j=i+1}^{n}(1-\alpha_j)$. Note that when $\alpha_i = \alpha$ for all $i$ the expression above reduces to
\[
	(1-\alpha)^n Q_1 + \alpha R_n + \sum_{i=1}^{n-1}\alpha(1-\alpha)^{n-i}R_i = (1-\alpha)^n Q_1 + \sum_{i=1}^{n}\alpha(1-\alpha)^{n-i}R_i,
\]
which is exactly (2.6).


\section{Exercise 2.5}

\textbf{Design and conduct an experiment to demonstrate the difficulties that sample-average methods have for nonstationary problems. Use a modified version of the 10-armed testbed in which all the $q_*(a)$ start out equal and then take independent random walks (say by adding a normally distributed increment with mean zero and standard deviation 0.01 to all the $q_*(a)$ on each step). Prepare plots like Figure 2.2 for an action-value method using sample averages, incrementally computed, and another action-value method using a constant step-size parameter, $\alpha = 0.1$. Use $\epsilon = 0.1$ and longer runs, say of 10,000 steps.}
\\ \\ 
As we can see from Figure \ref{fig:KArmedBanditNonStationarySampleAveraging}, the sample averaging method chooses the optimal action less than 50\% of the time, even after 10,000 time steps. Using a constant step-size of 0.1 works better, as shown in Figure \ref{fig:KArmedBanditNonStationaryConstantStep} where the optimal action is chosen around 75\% of the time after 10,000 time steps.  The sample averaging method does not work as well because after a significant number of time steps the step-size of $1/n$ becomes very small and the action-value estimates cannot adapt to the non-stationary environment.


\begin{figure}
	\centering
	\caption{Average performance of non-stationary $K$-armed bandit with sample averaging.}
	\includegraphics[scale=0.5]{ex2_5_1}
	\label{fig:KArmedBanditNonStationarySampleAveraging}
\end{figure}

\begin{figure}
	\centering
	\caption{Average performance of non-stationary $K$-armed bandit with constant step-size $\alpha=0.1$.}
	\includegraphics[scale=0.5]{ex2_5_2}
	\label{fig:KArmedBanditNonStationaryConstantStep}
\end{figure}


\section{Exercise 2.6: Mysterious Spikes}

\textbf{The results shown in Figure 2.3 should be quite reliable because they are averages over 2000 individual, randomly chosen 10-armed bandit tasks. Why, then, are there oscillations and spikes in the early part of the curve for the optimistic method? In other words, what might make this method perform particularly better or worse, on average, on particular early steps?}
\\ \\ 
On the 1st time step all arms have an estimated value of 5 and the optimistic method has to choose an action at random. As the $q_*(a)$ are selected from a normal distribution with mean 0 and variance 1, it is extremely unlikely that the reward will be greater than or equal to 5. Therefore, the estimated value of the chosen action will decrease and on the 2nd time step the algorithm will choose randomly between the remaining 9 actions. Again the reward will be less than 5 and the estimated value of this action will decrease and on the 3rd time step the algorithm will choose randomly between the remaining 8 actions. This will continue for the first ten time steps until all ten actions have been selected and their estimated values reduced. Thus, as the selection during the first ten steps is completely random, on each of these steps the optimal value is chosen 10\% of the time on average, as show in the graph.
\\ \\
Now on the 11th time step the action whose value was reduced the least, that is, the action which obtained the greatest reward during the first ten steps, will be selected. Over many runs the optimal action will return the highest reward and will therefore be selected most frequently on the 11th step. From the plot it looks like the first large spike occurs on the 11th step, where the optimal action is selected in just over 40\% of the runs. 
\\ \\
If the optimal action is selected on the 11th step then its estimated value will be reduced further such that it will not realistically be chosen again until after all other actions have also been sampled and had their estimated values reduced. Thus the next realistic opportunity for the optimal action to be selected will be on the 21st step, where there appears to be another spike in the graph up to just under 20\%. If on the other hand the optimal action is not selected on the 11th step then it is the most likely of the remaining nine actions to be selected on the 12th step, and so on. 
\\ \\
After thirty steps each action will likely have been chosen three times and the estimated values will have been reduced such that they are much closer to the true values. As a result, from this point on the spikes in the graph are noticeably smaller.  


\section{Exercise 2.7: Unbiased Constant-Step-Size Trick}

\textbf{In most of this chapter we have used sample averages to estimate action values because sample averages do not produce the initial bias that constant step sizes do (see the analysis leading to (2.6)). However, sample averages are not a completely satisfactory solution because they may perform poorly on nonstationary problems. Is it possible to avoid the bias of constant step sizes while retaining their advantages on nonstationary problems? One way is to use a step size of
\[
	\beta_n \doteq \alpha /  \bar{o}_n,
\]
to process the $n$th reward for a particular action, where $\alpha > 0$ is a conventional constant step size, and $\bar{o}_n$ is a trace of one that starts at 0:
\[
	\bar{o}_n \doteq \bar{o}_{n-1} + \alpha(1-\bar{o}_{n-1}), 	
\]
for $n\geq 0$ with $\bar{o}_0 \doteq 0$. Carry out an analysis like that in (2.6) to show that $Q_n$ is an exponential recency-weighted average \emph{without initial bias}.}
\\ \\ 
We have
\begin{align*}
	Q_{n+1} & = Q_n + \beta_n[R_n - Q_n] \\
	& = \beta_n R_n + (1-\beta_n) Q_n \\
	& = \beta_n R_n + (1-\beta_n)[\beta_{n-1} R_{n-1} + (1-\beta_{n-1}) Q_{n-1}] \\
	& = \beta_n R_n + (1-\beta_n)\beta_{n-1}R_{n-1} + (1-\beta_n)(1-\beta_{n-1})Q_{n-1} \\
	& = \frac{\alpha}{\bar{o}_n} R_n + \left(1-\frac{\alpha}{\bar{o}_n}\right)\frac{\alpha}{\bar{o}_{n-1}}R_{n-1} + \left(1-\frac{\alpha}{\bar{o}_n}\right)\left(1-\frac{\alpha}{\bar{o}_{n-1}}\right)Q_{n-1} \\
	& = \frac{\alpha}{\bar{o}_n} R_n + \left(\frac{\bar{o}_n - \alpha}{\bar{o}_n}\right)\frac{\alpha}{\bar{o}_{n-1}}R_{n-1} + \left(\frac{\bar{o}_n - \alpha}{\bar{o}_n}\right)\left(\frac{\bar{o}_{n-1} - \alpha}{\bar{o}_{n-1}}\right)Q_{n-1} \\
	& = \frac{\alpha}{\bar{o}_n} R_n + \left(\frac{\bar{o}_{n-1} + \alpha(1-\bar{o}_{n-1}) - \alpha}{\bar{o}_n}\right)\frac{\alpha}{\bar{o}_{n-1}}R_{n-1} + \left(\frac{\bar{o}_{n-1} + \alpha(1-\bar{o}_{n-1}) - \alpha}{\bar{o}_n}\right)\left(\frac{\bar{o}_{n-1} - \alpha}{\bar{o}_{n-1}}\right)Q_{n-1} \\
	& = \frac{\alpha}{\bar{o}_n} R_n + \left(\frac{\bar{o}_{n-1}(1 - \alpha)}{\bar{o}_n}\right)\frac{\alpha}{\bar{o}_{n-1}}R_{n-1} + \left(\frac{\bar{o}_{n-1}(1 - \alpha)}{\bar{o}_n}\right)\left(\frac{\bar{o}_{n-1} - \alpha}{\bar{o}_{n-1}}\right)Q_{n-1} \\	
	& = \frac{\alpha}{\bar{o}_n} R_n + \frac{(1-\alpha)\alpha}{\bar{o}_n} R_{n-1} + \frac{(1-\alpha)(\bar{o}_{n-1} - \alpha)}{\bar{o}_n}	Q_{n-1}.		
\end{align*}
Now 
\begin{align*}
	(\bar{o}_{n-1} - \alpha) Q_{n-1} & = (\bar{o}_{n-1} - \alpha) [\beta_{n-2} R_{n-2} + (1-\beta_{n-2}) Q_{n-2}] \\
	& = (\bar{o}_{n-1} - \alpha)\beta_{n-2} R_{n-2} + (\bar{o}_{n-1} - \alpha) (1-\beta_{n-2}) Q_{n-2}	\\
	& = (\bar{o}_{n-1} - \alpha)\frac{\alpha}{\bar{o}_{n-2}} R_{n-2} + (\bar{o}_{n-1} - \alpha)	\left(1-\frac{\alpha}{\bar{o}_{n-2}}\right) Q_{n-2} \\
	& = (\bar{o}_{n-2} + \alpha(1 - \bar{o}_{n-2}) - \alpha)\frac{\alpha}{\bar{o}_{n-2}} R_{n-2} + (\bar{o}_{n-2} + \alpha(1 - \bar{o}_{n-2}) - \alpha)	\left(1-\frac{\alpha}{\bar{o}_{n-2}}\right) Q_{n-2} \\
	& = \bar{o}_{n-2}(1 - \alpha)\frac{\alpha}{\bar{o}_{n-2}} R_{n-2} + \bar{o}_{n-2}(1 - \alpha) \left(\frac{\bar{o}_{n-2} - \alpha}{\bar{o}_{n-2}}\right) Q_{n-2} \\
	& = (1 - \alpha)\alpha R_{n-2} + (1 - \alpha)(\bar{o}_{n-2} - \alpha)  Q_{n-2} \\
\end{align*}
Hence
\begin{align*}
	Q_{n+1} & = \frac{\alpha}{\bar{o}_n} R_n + \frac{(1-\alpha)\alpha}{\bar{o}_n} R_{n-1} + \frac{(1-\alpha)}{\bar{o}_n}[(1 - \alpha)\alpha R_{n-2} + (1 - \alpha)(\bar{o}_{n-2} - \alpha)  Q_{n-2}] \\
	& =  \frac{\alpha}{\bar{o}_n} R_n + \frac{(1-\alpha)\alpha}{\bar{o}_n} R_{n-1} + \frac{(1-\alpha)^2\alpha}{\bar{o}_n} R_{n-2} + \frac{(1-\alpha)^2(\bar{o}_{n-2} - \alpha)}{\bar{o}_n}	Q_{n-2} \\
	& = \frac{\alpha}{\bar{o}_n} R_n + \frac{(1-\alpha)\alpha}{\bar{o}_n} R_{n-1} + \frac{(1-\alpha)^2\alpha}{\bar{o}_n} R_{n-2} + \dots + \frac{(1-\alpha)^{n-1}\alpha}{\bar{o}_n} R_1 + \frac{(1-\alpha)^{n-1}(\bar{o}_1 - \alpha)}{\bar{o}_n}Q_1 \\
	& = \frac{(1-\alpha)^{n-1}(\bar{o}_1 - \alpha)}{\bar{o}_n}Q_1 + \frac{\alpha}{\bar{o}_n} \sum_{i=1}^{n} (1-\alpha)^{n-i} R_i \\
	& = \frac{(1-\alpha)^{n-1}(\bar{o}_0 + \alpha(1 - \bar{o}_0) - \alpha)}{\bar{o}_n}Q_1 + \frac{\alpha}{\bar{o}_n} \sum_{i=1}^{n} (1-\alpha)^{n-i} R_i \\
	& = \frac{(1-\alpha)^{n-1}(\alpha - \alpha)}{\bar{o}_n}Q_1 + \frac{\alpha}{\bar{o}_n} \sum_{i=1}^{n} (1-\alpha)^{n-i} R_i \\
	& = \frac{\alpha}{\bar{o}_n} \sum_{i=1}^{n} (1-\alpha)^{n-i} R_i.
\end{align*}
This shows that $Q_n$ is an exponential recency-weighted average where the weight given to $R_i$ decays exponentially according to the exponent on $1-\alpha$. Also, as the average does not depend on $Q_1$ it has no initial bias.


%\begin{align*}	
%	Q_{n+1} & = Q_1\prod_{j=1}^{n}(1-\beta_j) + \beta_n R_n + \sum_{i=1}^{n-1}\beta_i R_i \prod_{j=i+1}^{n}(1-\beta_j) \\
%			& = Q_1\prod_{j=1}^{n}(1-\frac{\alpha}{\bar{o}_j}) + \frac{\alpha}{\bar{o}_n} R_n + \sum_{i=1}^{n-1}\frac{\alpha}{\bar{o}_i} R_i \prod_{j=i+1}^{n}(1-\frac{\alpha}{\bar{o}_j}) \\
%			& = Q_1\prod_{j=1}^{n}(1-\frac{\alpha}{\bar{o}_j}) + \frac{\alpha}{\bar{o}_n} R_n + \sum_{i=1}^{n-1}\frac{\alpha}{\bar{o}_i} R_i \prod_{j=i+1}^{n}(1-\frac{\alpha}{\bar{o}_j})
%\end{align*}


\end{document}